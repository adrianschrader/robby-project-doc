\documentclass[a4paper, 10pt, twocolumn]{scrartcl}
\usepackage[ngerman]{babel}

% Mathematics
\usepackage{amsmath}
\usepackage{amssymb}
\numberwithin{equation}{subsection}

% Font and Style
\usepackage{ucs}
\usepackage[T1]{fontenc}
\usepackage{abstract}
\usepackage{url}
\usepackage{lipsum}
\usepackage{hyperref}
\hypersetup{colorlinks=false}

% Source Code Highlighting
\usepackage{caption}
\usepackage[section]{minted}
\usemintedstyle{trac}
\setminted[java]{autogobble,mathescape,fontfamily=courier,breaklines,fontsize=\footnotesize}
\renewcommand{\listingscaption}{Quellcode}

% Font Package (select one)
%\usepackage{lmodern}
%\usepackage{kpfonts}
%\usepackage{ccfonts}
%\usepackage[light]{cmbright}
%\usepackage{antpolt}
\usepackage[]{iwona}

% Bibliography
\usepackage[numbers,round]{natbib}
\usepackage[babel,german=guillemets]{csquotes}
\bibliographystyle{alphadin}

% Image and Graphics
\usepackage{graphicx}
\usepackage{dblfloatfix}
\usepackage[left=2.3cm, right=2.3cm, top=2.5cm, bottom=4.5cm]{geometry}
\setlength{\columnsep}{15pt}
\setlength{\columnseprule}{0.0pt}

% Document Information
\title{Robby Projekt}
\subtitle{Erweiterung des Greenfoot Roboter-Szenarios um Sensorik, Speicher und  Hindernisumgehung}
\author{Adrian Schrader \and Moritz Jung \and Alexander Riecke}
\date{\today}

\begin{document}
  % Create Title, Abstract and TOC
  \twocolumn[\maketitle \begin{abstract} \begin{minipage}{1.0\linewidth}
  \input{tex/abstract}
  \vspace{0.8cm}
  \end{minipage}\end{abstract} ]

  \tableofcontents

  % CONTENT
  \section{Sensorik}

\subsection{Aufgabenstellung}
\label{snsk:aufgabe}
Um Robby Anhaltspunkt für seine Aktionen zu geben, sollen zwei verschiedene Arten von Sensoren eingeführt werden, mit denen Robby seine Umgebung abtasten kann. Robby kann nicht durch eine Wand laufen, also sollte er erkennen können, ob in seiner Umgebung solch ein Hindernis auftaucht.

Eines der Hauptaktionen eines Roboters in diesem Szenario ist das Sammeln von Akkus für Energie. Um gezielter nach Akkus suchen muss Robby seine Umgebung nach ihnen abtasten.

\paragraph{wandHinten()}
Wenn, in Bezug auf Robbys Laufrichtung gesehen, ein Actor der Klasse Wand ein Feld hinter Robby steht, soll die Methode \mintinline{java}{true} zurückgeben, ansonsten \mintinline{java}{false}.

\paragraph{akkuVorne() [ akkuRechts(), akkuLinks() ]}
Wenn, in Bezug auf Robbys Laufrichtung gesehen, ein Actor der Klasse Akku ein Feld vor [rechts von, links von] Robby steht, soll die Methode \mintinline{java}{true} zurückgeben, ansonsten \mintinline{java}{false}.

\subsection{Problematik}
\label{snsk:problem}
Diese vier Fälle können auf das Problem reduziert werden aus der Blickrichtung des Roboters und dem spezifischen Suchwinkel einen Vektor vom Roboter zum Suchfeld zu konstruieren, damit überprüft werden kann, ob die Methode \mintinline{java}{this.getOneObjectAtOffset(int v_x,int v_y,Class<?> c)} ein Objekt übergibt oder nicht.
Die Mutterklasse Roboter löst die Aufgabe, in dem sie jeden einzelnen Suchvektor als einzelne Methode implementiert und in ihr die vier Blickrichtungen abfragt, um daraus einen fest einprogrammierten Vektor auszuwählen.
Diese Herangehensweise funktioniert zwar, ist jedoch für eine schlanke, wiederverwendbare, nachvollziehbare und skalierbare Klasse nicht geeignet.
Um die Klasse evtl. später um Abfragen zusätzlicher Aktoren erweitern zu können, muss die Abfrage in einer einzigen Methode stattfinden. Diese errechnet dynamisch aus den Faktoren den gewünschten Vektor.

\subsection{Lösung}
\label{snsk:loesung}
Um die Lösung für dieses Problem zu verstehen ist es hilfreich den gesuchten Vektor $ \vec{v} $ als Zeiger zu verstehen, dessen Betrag immer auf $ \left| \vec{v} \right| = 1 $ genormt ist. Aus dem Winkel $ \theta $ von der Horizontalen lässt sich dann die Komponente in x und y-Richtung mithilfe von Sinus und Kosinus errechnen.
\begin{align}
  v_x = \left| \vec{v} \right| \cdot cos (\theta) && v_y = \left| \vec{v} \right| \cdot sin (\theta)
\end{align}

Für unseren Anwendungsfall interessieren uns nur ganzzahlige Werte von $v_x$ und $v_y$ zwischen -1 und 1. Daher können wir die Domäne für $\theta$ enger eingrenzen.

\begin{align}
  \theta \in \Big\{ k \cdot \frac{\pi}{2} \, | \, k \in \mathbb{N}_0 \Big\}
\end{align}

Für die in Abschnitt \ref{snsk:problem} besprochene Methode müssen wir jedoch zuerst den Winkel für die Laufrichtung und den Suchwinkel addieren, um auf den gesuchten Vektor zu kommen. Dieser soll dann in das Bogenma\ss  umgerechnet und zu den Komponenten verarbeitet werden. Über eine Abfrage der Methode \mintinline{java}{this.getOneObjectAtOffset(int v_x,int v_y,Class<?> c)} lässt sich dann der überprüfen, ob der gesuchte Actor existiert oder nicht.
\begin{listing}
  \begin{minted}{java}
    /**
     * Der Sensor überprüft, ob sich neben der Laufrichtung von Robby ein
     * anderer Actor befindet.
     * @param direction Winkel von der Laufrichtung zum Suchfeld.
     * @param class Klasse des gesuchten Actors
     * @return boolean
     */
    public boolean istObjektNebendran(int direction, Class<?> cl)
    {
       double angle = (this.getRotation() + direction) / 180.0 * Math.PI;

       return (this.getOneObjectAtOffset(
              (int)Math.cos(angle),
              (int)Math.sin(angle), cl) != null);
    }
  \end{minted}
  \caption{Die implementierte Methode \mintinline{java}{istObjektNebendran(int,Class<?>)} aus Robby.java }
\end{listing}

  \subsection{Speicher}

\subsubsection*{Aufgabenstellung}
Die Klasse Robby sollte durch die in ihrer Funktion erweiterten Methoden akkuAufnehmen() und schraubeAblegen() mit zwei neu instanziierten globalen Variablen anzahlAkkus und anzahlSchrauben bestimmte Änderungen abspeichern können.
Hierzu war vorgesehen, dass die Klasse Robby die vorher festgelegte Anzahl an Schrauben und Akkus nach Ausführung der Methoden jeweils um eins erniedrigt bzw. um eins erhöht. Sollten die Methoden nicht ausführbar sein, was durch ein leeres Feld ohne Akku oder zu wenig Schrauben verursacht werden könnte, war eine aussagekräftige Meldung vorgesehen, die dies beschreibt.


\subsubsection*{Problematik}
Die ersten Schritte zur Herangehensweise an die Aufgabe waren zunächst die globalen Variablen. Anhand der vorgebenen Werte beider Variabeln, entstand hier bereits die Idee bei einer if-Abfrage diese Werte als Bedingung für das weitere Verfahren in der Methode zu verwenden.
Mit Hilfe dieser Grundüberlegung entwickelten sich beide Methoden in der Planung zu jeweils einer einzigen if-Abfrage, in der mehre Bedingungen und Szenarien gleichzeitig abgedeckt werden. So war zum Beispiel vorgesehen in einer if-Abfrage ein leeres Feld UND eine noch nicht überschrittene Maximalanzahl an Akkus zu implementieren.

Dadurch entstand jedoch nach einigen Testdurchläufen ein Problem mit der getrennten Ausgabe der Fehlermeldung und der getrennten, nacheinander abfolgenden Ausführung in der if-Schleife.

\subsubsection*{Lösung}
Hierfür war dann eine Verschachtelung der Befehlskette vorgesehen.
Hierzu wurden dann die Befehle mit zweiten if-Abfragen ineinander verschachtelt und es entstand eine chronologische Vorgehensweise.
Indem die Klasse Robby an dem bereits oben erwähnten Beispiel zunächst das Feld überprüft und dort einen Akku erkennt (Feld nicht leer), wird nun erst die Maximalanzahl von zehn Akkus überprüft. Ist diese auch noch nicht überschritten wird nun ein Akku aufgenommen und in den Speicher einbezogen. Bei Überlastung der Speichergrenze (grö\ss{}er 10) wird nun in der gleichen Abfrage per else-Schleife eine Fehlermeldung ausgegeben.
Falls sich auf dem überprüften Feld jedoch kein Akku befindet wird hier jetzt in der äu\ss{}eren if-Schleife separat per else-Abfrage eine entsprechende Fehlermeldung gezeigt, was das aufgetretene Problem letztendlich gelöst hat.

Ein Test der beiden erweiterten Methoden zeigt nun eine erwartete Verringerung der Schrauben- und Erhöhung der Akku-Anzahl bezogen auf die Werte in den globalen Variablen. In entsprechenden Szenarien reagiert die Klasse Robby auch mit den Situationen entsprechenden Fehlermeldungen, wodurch die Speicherfunktion ihren Anforderungen entsprechend erfolgreich funktioniert.

  \subsection{Hindernisse}

\subsubsection*{Aufgabenstellung}
Aufgabenstellung war es, Robby ein geschlossenes Hindernis aus Wänden umrunden zu lassen, so dass er nach erfolgreicher Umrundung wieder am Ausgangspunkt ankommt.
Auch die Weltgrenze wird als Hindernis wahrgenommen

\subsubsection*{Problematik}
Als augenscheinlich einfachste Lösung stellte sich eine Verkettung von mehreren if-Abfragen heraus, in der nacheinander von Rechts beginnend gegen den Uhrzeigersinn die einzelnen möglichen vewegungsrichtungen abgefragt wurden. Diese Lösung erwies sich jedoch als sehr verschachtelt, da für jede Abfrage eine einzelne Reaktionsanweisung erstellt werden musste.
Weiterhin konnten anfangs nur sehr begrenzte Tests durchgeführt werden, da die Funktion in einem Ablauf ausgeführt wird und somit keine Möglichkeit besteht, auch weitergehende Aufgaben währenddessen auszuführen.


\subsubsection*{Lösung}
Durch Einführen einer do-while-Schleife und einer for-Schleife für die Abfragen der Umgebung, also der Wände und Weltgrenzen auf benachbarten Feldern, konnte der Code auf ein Minimum reduziert werden, in dem  zwar immernoch von Rechts beginnend die Umgebung angefragt wird, jedoch die Funktion immer wieder mit anderen Winkel aufgerufen wird und nicht für jede Richtung eine neue Abfrage geschrieben werden muss.
Auch eine Möglichkeit zum Ausführen von weiterem Code während der Umrundung wurde Implementiert, um beispielsweise auch weitergehende Aufgaben, wie zum Beispiel das Ablegen von Schrauben zu ermöglichen, aber auch , um zu die zu Testzwecken benötigten Daten während der Umrundung zu erfassen.


  % Bibliography
  \bibliography{bib/bibliography.bib}

  \onecolumn{
    \captionof{listing}{Vollständige, dokumentierte Version der Klasse Robby.java, die alle oben beschreibenen Zusatzfuntionen enthält. }
    \inputminted[firstline=1,linenos,numbers=left]{java}{code/Robby.java}
  }


\end{document}
