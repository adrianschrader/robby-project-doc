\section{Sensorik}

\subsection{Aufgabenstellung}
\label{snsk:aufgabe}
Um Robby Anhaltspunkt für seine Aktionen zu geben, sollen zwei verschiedene Arten von Sensoren eingeführt werden, mit denen Robby seine Umgebung abtasten kann. Robby kann nicht durch eine Wand laufen, also sollte er erkennen können, ob in seiner Umgebung solch ein Hindernis auftaucht.

Eines der Hauptaktionen eines Roboters in diesem Szenario ist das Sammeln von Akkus für Energie. Um gezielter nach Akkus suchen muss Robby seine Umgebung nach ihnen abtasten.

\paragraph{wandHinten()}
Wenn, in Bezug auf Robbys Laufrichtung gesehen, ein Actor der Klasse Wand ein Feld hinter Robby steht, soll die Methode \mintinline{java}{true} zurückgeben, ansonsten \mintinline{java}{false}.

\paragraph{akkuVorne() [ akkuRechts(), akkuLinks() ]}
Wenn, in Bezug auf Robbys Laufrichtung gesehen, ein Actor der Klasse Akku ein Feld vor [rechts von, links von] Robby steht, soll die Methode \mintinline{java}{true} zurückgeben, ansonsten \mintinline{java}{false}.

\subsection{Problematik}
\label{snsk:problem}
Diese vier Fälle können auf das Problem reduziert werden aus der Blickrichtung des Roboters und dem spezifischen Suchwinkel einen Vektor vom Roboter zum Suchfeld zu konstruieren, damit überprüft werden kann, ob die Methode \mintinline{java}{this.getOneObjectAtOffset(int v_x,int v_y,Class<?> c)} ein Objekt übergibt oder nicht.
Die Mutterklasse Roboter löst die Aufgabe, in dem sie jeden einzelnen Suchvektor als einzelne Methode implementiert und in ihr die vier Blickrichtungen abfragt, um daraus einen fest einprogrammierten Vektor auszuwählen.
Diese Herangehensweise funktioniert zwar, ist jedoch für eine schlanke, wiederverwendbare, nachvollziehbare und skalierbare Klasse nicht geeignet.
Um die Klasse evtl. später um Abfragen zusätzlicher Aktoren erweitern zu können, muss die Abfrage in einer einzigen Methode stattfinden. Diese errechnet dynamisch aus den Faktoren den gewünschten Vektor.

\subsection{Lösung}
\label{snsk:loesung}
Um die Lösung für dieses Problem zu verstehen ist es hilfreich den gesuchten Vektor $ \vec{v} $ als Zeiger zu verstehen, dessen Betrag immer auf $ \left| \vec{v} \right| = 1 $ genormt ist. Aus dem Winkel $ \theta $ von der Horizontalen lässt sich dann die Komponente in x und y-Richtung mithilfe von Sinus und Kosinus errechnen.
\begin{align}
  v_x = \left| \vec{v} \right| \cdot cos (\theta) && v_y = \left| \vec{v} \right| \cdot sin (\theta)
\end{align}

Für unseren Anwendungsfall interessieren uns nur ganzzahlige Werte von $v_x$ und $v_y$ zwischen -1 und 1. Daher können wir die Domäne für $\theta$ enger eingrenzen.

\begin{align}
  \theta \in \Big\{ k \cdot \frac{\pi}{2} \, | \, k \in \mathbb{N}_0 \Big\}
\end{align}

Für die in Abschnitt~\ref{snsk:problem} besprochene Methode müssen wir jedoch zuerst den Winkel für die Laufrichtung und den Suchwinkel addieren und in das Bogenma\ss umrechnen. Aufgespalten in seine Komponenten kann der Vektor in die Methode \mintinline{java}{this.getOneObjectAtOffset(int v_x,int v_y,Class<?> c)} eingegeben und damit die Existenz des Objekts überprüft werden. Dargestellt ist die implementierte Methode im Quellcode~\ref{lst:istObjektNebendran}.
\begin{listing}
  \begin{minted}{java}
    /**
     * Der Sensor überprüft, ob sich neben der Laufrichtung von Robby ein
     * anderer Actor befindet.
     * @param direction Winkel von der Laufrichtung zum Suchfeld.
     * @param class Klasse des gesuchten Actors
     * @return boolean
     */
    public boolean istObjektNebendran(int direction, Class<?> cl)
    {
       double angle = (this.getRotation() + direction) / 180.0 * Math.PI;

       return (this.getOneObjectAtOffset(
              (int)Math.cos(angle),
              (int)Math.sin(angle), cl) != null);
    }
  \end{minted}
  \caption{Implementation der Basismethode für die Sensorik aus Robby.java. }
  \label{lst:istObjektNebendran}
\end{listing}

Nachdem wir diese Grundmethode implementiert haben, können die gesuchten Methoden durch eine einzelne Abfrage dargestellt werden. \mintinline{java}{akkuVorne()} verweist bspw. auf  \mintinline{java}{istObjektNebendran(0, Akku.class)}. Da Greenfoot Winkel im Uhrzeigersinn misst, verwendet die Abfrage von \mintinline{java}{akkuRechts()} den Suchwinkel $90^\circ$. Eine Liste aller Implementierungen ist in Quellcode~\ref{lst:robby} zu finden.
