\section{Gruppenarbeit}
Aus den Funktionsanforderungen des Lastenhefts für die Klasse Robby und die Dokumentation haben wir kleinere Aufgabenpakete zusammengestellt, die in der Gruppe verteilt werden können.

\begin{table*}
    \centering
    \caption{Aufgabenübersicht und Zuständigkeiten des Projekts}
    \begin{tabular}{ l l l }
        \textbf{Bereich} & \textbf{Aufgabe} & \textbf{Zuständigkeit}\\ \hline\hline
        Sensorik & Funktionalität implementieren & Adrian \\
                 & Sourcecode kommentieren & Adrian \\
                 & Sourcecode aufräumen, anpassen und zusammenfügen & Adrian \\
                 & Schriftliche Dokumentation & Adrian \\ \hline
        Speicher & Funktionalität implementieren & Alex \\
                 & Sourcecode kommentieren & Alex \\
                 & Sourcecode aufräumen, anpassen und zusammenfügen & Adrian \\
                 & Schriftliche Dokumentation & Alex \\ \hline
       Hindernis & Funktionalität implementieren & Moritz \\
                 & Sourcecode kommentieren & Moritz \\
                 & Sourcecode aufräumen, anpassen und zusammenfügen & Adrian / Moritz \\
                 & Schriftliche Dokumentation & Moritz \\ \hline
        Tests    & Planung & Adrian \\
                 & Funktionalität implementieren & Adrian \\
                 & Sourcecode kommentieren & Adrian \\
                 & Sourcecode aufräumen, anpassen und zusammenfügen & Adrian \\
                 & Schriftliche Dokumentation & Adrian \\ \hline
    \end{tabular}
\end{table*}

\subsection*{Sourcecode-Verwaltung}
Um den Überblick über den aktuellen Stand zu behalten und die Sourcecodeversionen koordinert zusammenführen zu können, haben wir gehostete git-Repositorys mit Issuetrackern und Milestones eingerichtet. Nach Abgabe der Dokumentation sind diese auch öffentlich verfügbar. Das Robbyprojekt ist unter \url{https://github.com/adrianschrader/robby-project} und die Dokumentation als LaTex-Projekt unter \url{https://github.com/adrianschrader/robby-project-doc} zu finden.
