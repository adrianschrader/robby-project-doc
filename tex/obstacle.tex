\subsection{Hindernisse}

\subsubsection*{Aufgabenstellung}
Aufgabenstellung war es, Robby ein geschlossenes Hindernis aus Wänden umrunden zu lassen, so dass er nach erfolgreicher Umrundung wieder am Ausgangspunkt ankommt.
Auch die Weltgrenze wird als Hindernis wahrgenommen

\subsubsection*{Problematik}
Als augenscheinlich einfachste Lösung stellte sich eine Verkettung von mehreren if-Abfragen heraus, in der nacheinander von Rechts beginnend gegen den Uhrzeigersinn die einzelnen möglichen vewegungsrichtungen abgefragt wurden. Diese Lösung erwies sich jedoch als sehr verschachtelt, da für jede Abfrage eine einzelne Reaktionsanweisung erstellt werden musste.
Weiterhin konnten anfangs nur sehr begrenzte Tests durchgeführt werden, da die Funktion in einem Ablauf ausgeführt wird und somit keine Möglichkeit besteht, auch weitergehende Aufgaben währenddessen auszuführen.


\subsubsection*{Lösung}
Durch Einführen einer do-while-Schleife und einer for-Schleife für die Abfragen der Umgebung, also der Wände und Weltgrenzen auf benachbarten Feldern, konnte der Code auf ein Minimum reduziert werden, in dem  zwar immernoch von Rechts beginnend die Umgebung angefragt wird, jedoch die Funktion immer wieder mit anderen Winkel aufgerufen wird und nicht für jede Richtung eine neue Abfrage geschrieben werden muss.
Auch eine Möglichkeit zum Ausführen von weiterem Code während der Umrundung wurde Implementiert, um beispielsweise auch weitergehende Aufgaben, wie zum Beispiel das Ablegen von Schrauben zu ermöglichen, aber auch , um zu die zu Testzwecken benötigten Daten während der Umrundung zu erfassen.
